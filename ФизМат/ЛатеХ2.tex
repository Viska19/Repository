%Шаблон статьи для сборника ОГУ
\documentclass[a4paper]{article}
\usepackage{sbornik} % стилевой файл сборника, находится в архиве и должен располагаться в одном каталоге с Вашей статьей
\begin{document}
% стоки обязательно оставить
\renewcommand{\refname}{\centering \textnormal {\bf ЛИТЕРАТУРА}}
\renewcommand{\figurename}{Рисунок}
% стоки обязательно оставить


%Начало работы с текстом статьи

% УДК для Вашей статьи --- обязательно
\udk{УДК 004.912}
%начало оформление шапки статьи

\begin{center}
%Название Вашей статьи --- {} -- обязательное поле; [] необязательное, например, ссылка на грант или благодарность
\ttitle{ИНФОРМАЦИОННЫЕ ОСНОВЫ ВЫДЕЛЕНИЯ АББРЕВИАТУР И ИХ РАСШИФРОВКИ В ТЕКСТЕ НА РУССКОМ ЯЗЫКЕ
}
%Сведения об авторе (ах) --- {}-обязательные поля; []-необязательные
\avtor{И.\,И.,~Ивнов }{В.\,В.,~Васильев }{П.\,П.~Петров}{\parФГКВОУВО «Академия ФСО России»,}{\parРоссия, Орёл
}
\end{center}
%Аннотация и ключевые слова на русском языке
\ruAbstract{Аннотация представляет собой краткую характеристику
документа с точки зрения его назначения, содержания, вида, формы и
других особенностей. Аннотация включает характеристику основной
темы, исследуемой проблемы, цели работы и ее результаты. В
аннотации указывают, что нового несет в себе данный документ в
сравнении с другими, родственными по тематике и целевому
назначению.}

\ruKey{ аббревиатура, расшифровка аббревиатуры, выделение аббревиатур, классификация аббревиатур.}


\textbf{1.Вводные положения}
\parРазвитие аббревиации и использование сокращенных лексических единиц – общая тенденция для многих алфавитных языков. Так, аббревиатуры широко используются не только в специализированных областях знания, но и в повседневной коммуникации [1]. 
\parВведём ряд определений. Под \emph{аббревиатурой} будем понимать слово, образованное сокращением слова или словосочетания, читаемое по названию начальных букв или по начальным и крайним (общепринятые аббревиатуры) звукам слов, входящих в него. Под \emph{расшифровкой} или \emph{полной формой аббревиатуры} будем понимать последовательность слов, от которых образована аббревиатура. Введением аббревиатуры в текст является определенная последовательность аббревиатуры и её расшифровки в одном предложении текста.\emph{Аббревиатурой без расшифровки} является аббревиатура, не имеющая расшифровки в предложении, где она расположена. Под \emph{выделением аббревиатуры} (расшифровки) будем понимать получение структурной информации об аббревиатуре (расшифровке) для её дальнейшего использования.
\parУпотребление аббревиатур – специфическая особенность научно-технических текстов, в которых аббревиатурам принадлежит большая доля информационной нагрузки [2]. В текстах художественного стиля практически отсутствуют аббревиатуры в виду отсутствия ёмких терминов, которые необходимо сокращать.
\parВ научно-технических текстах на русском языке из различных областей знаний используются разнообразные аббревиатуры, что затрудняет возможность интуитивной расшифровки аббревиатуры человеком, читающим текст. При отсутствии \emph{списка аббревиатур и} соответствующих им \emph{расшифровок} (САиР) для текста возникают трудности с интерпретацией аббревиатур. При этом в большинстве текстов достаточно информации для того, чтобы по определенным признакам восстановить или создать данный список. Это утверждение положено в основу данной статьи. Для отдельных текстов информации, содержащейся внутри них, может быть недостаточно, чтобы восстановить САиР. Тогда следует прибегать к анализу других текстов схожей тематики.
\parПотребность в восстановлении САиР может возникнуть при решении широкого класса задач обработки текстов. Здесь можно упомянуть межъязыковые преобразования текстов, в том числе – конверсию графических систем письма [3], расчет характеристик сложности текста [4], автоматизированное извлечение ключевых слов [5], рерайтинг [6] и квалиметрический анализ текста [7]. Во всех перечисленных задачах необходимо произвести предварительное выделение аббревиатур и их расшифровок (ВАиР) из текста.
\parИсходя из результатов исследования тексты на русском и английском языках можно разделить на три типа (рисунок 1):
\par1. Тексты, в которых присутствует САиР, \emph{введения аббревиатур и аббревиатуры без расшифровок}.
\par2.Тексты, в которых присутствуют  \emph{введения аббревиатур и аббревиатуры без расшифровок} (введённые ранее).
\par3.Тексты, которых присутствуют только  \emph{аббревиатуры без расшифровок} (ранее не были введены).
\parПредполагается, что результаты исследования будут актуальны и для других алфавитных языков. Также установлено, что тексты научно-технического стиля более насыщены аббревиатурами, чем тексты аналогичного объёма художественного стиля.
\par\textbf{2.Состояние предметной области}
\parВ работах [8-10] представляются различные классификации аббревиатур, но не предложено решений по выделению их из текстов.
\parВ работе [11] предложено решение для нахождения полного названия журнала по его аббревиатуре. Неизвестная аббревиатура, указанная пользователем, приводилась в формат регулярного выражения, которое предполагало возможный набор слов, начинающийся с указанных букв. Специфическая реализация полученного решения не позволяет использовать его для определения полных форм аббревиатур из других предметных областей.
\parВ работах [12, 13] на основании определения частот встречаемости соседних слов определяется мера их связности, что позволяет предложить вероятные полные формы аббревиатур. Достоинством такого метода является его универсальность, недостатком – высокая трудоёмкость. Материалы данных работ использовались при разработке модели процесса выделения аббревиатур.
\parВ работах [14, 15] предложено исходный текст представлять в виде совокупности тем, которые образуются множеством входящих в них с разной вероятностью слов. Найденная схожесть частей текста используется как представление полной формы аббревиатуры. Данных подход предлагает множество решений с близкими вероятностями, что предусматривает дополнительную работу для пользователя на стадии отбора расшифровки интересующей аббревиатуры.
\parСуществуют программы для ЭВМ, зарегистрированные в Федеральной службе по интеллектуальной собственности (Роспатент), обладающие возможностью выявления САиР. Так, программа [16] реализует функцию автоматизированного формирования перечня аббревиатур, решает задачу формирования единой базы терминов (аббревиатур) и их определений (расшифровок). Программа [17] предназначена для автоматизированного извлечения терминологических структур из монографии заданной предметной области. Одной из основных функций программы является извлечение терминов, в частности, расшифровка аббревиатур.
\par\textbf{3.Классификация аббревиатур}
\parФормирование классификации аббревиатур осложнено особенностями их структуры, большой вариативностью, множеством различных способов аббревиации, а также взаимодействием аббревиации с другими способами словообразования. Исследователи [10, 18, 19] сходятся во мнении, что аббревиатуры можно подразделять на инициальные, сложносокращённые и общепринятые. В первом случае аббревиатура составляется из первых букв её расшифровки. Во втором случае в аббревиатуру включены не только первые, но и другие буквы сокращаемых слов [20]. В третьем случае аббревиатуры имеют уникальное представление в тексте и единственную расшифровку. Общепринятые аббревиатуры, как правило, интуитивно понятны и употребляются перед определёнными структурами в тексте.
\parДля решения задачи автоматизированного ВАиР из текста введём классификацию по структурно-информационным признакам, а также приведем в первом приближении их распространенность, изученную на материале ста случайно отобранных статей с ресурса Cyberleninka.ru. Аббревиатуры разделяются на три класса: инициальные, общепринятые и сложносокращённые. Инициальные и общепринятые аббревиатуры имеют выраженную структуру (прописные буквы, знаки препинания), которой сложносокращённые не обладают (структурный признак). При этом, общепринятые аббревиатуры имеют интуитивно понятный смысл, а инициальные требуют расшифровки в тексте (информационный признак). Сложносокращённые аббревиатуры, не имеющие в составе прописных букв (завхоз, ликбез и т.д.), рассматриваться в данной статье не будут. Инициальные аббревиатуры разделены на пять типов, каждый из которых отличается по структурным признакам (рисунок 2).
\parДля создания программного средства ВАиР необходимо учитывать особенности каждого класса рассматриваемых аббревиатур. 
\parОсобенности инициальных и общепринятых аббревиатур:
\par1.(тип А, 53) Инициальная аббревиатура, в которой слова полной формы разделены только пробелами и в неё входят только первые буквы слов полной формы. Например: центр информационной безопасности (ЦИБ), Latent Dirichlet Allocation (LDA).
\par2. (тип B, 5) Инициальная аббревиатура, в которой некоторые слова полной формы объединены знаком дефис или символом «косой черты». Например: оптико-тепловизионный комплекс (ОТК), read-only memory (ROM), input/output (IO).
\par3. (тип C, 22) Инициальная аббревиатура с элементами сложносокращённых слов. При этом, аббревиатура может состоять не только из прописных букв, но первая буква полной формы должна соответствовать первой букве аббревиатуры. Количество слов в расшифровке не совпадает с количеством букв в аббревиатуре. Например: гидрометеорологическая станция (ГМС), ammonium bifluoride (ABF), Белорусский автомобильный завод (БелАЗ), временно исполняющий обязанности (ВрИО).
\par4. (тип D, 5) Инициальная аббревиатура, отличная от языка документа. Например, протокол передачи файлов (FTP), временный идентификационный номер подвижного абонента (TMSI).
\par5. (тип E, 2) Инициальная аббревиатура, в которой буквы аббревиатуры разделены точками, а первые буквы слов полной формы соответствуют буквам в аббревиатуре. Например: Фамилия Имя Отчество (Ф.И.О.), Петроградская сторона (П.С.).
\par6. Общепринятые аббревиатуры (13), которые применяются в разных областях: адреса (г., ул., д., пр-т), звания (к-т, л-т), точные науки (см, Гц), время суток (a.m, p.m), элемент текста (P.S.) и т.п. Они не имеют полной формы в тексте и будут расшифровываться по словарю.
Для примеров были использованы аббревиатуры на русском и английском языках, но предполагается, что данная классификация актуальна и для других алфавитных языков.

\par\textbf{4.Модель процесса выделения аббревиатур и расшифровок из текста}
\parПроцесс выделения аббревиатур и их расшифровок в общем может состоять из двух этапов.
\par\textbf{Первый этап} заключается в разделении исходного текста на предложения. Он необходим для более точного определения расшифровок аббревиатур. \parРазделителем предложений в тексте могут являться восклицательные и вопросительные знаки, многоточия, знаки переноса строки и точки. Однако возникает ряд проблем, связанных с тем, что точки ставятся в тексте не только в конце предложения. Чаще всего точки можно встретить в следующих конструкциях: в датах (25.10.20 г.), в адресах (ул. Ленина, д. 7), в общих аббревиатурах (т.д.), в буквенно-цифровых обозначениях (66.КП.ВРБ.00.00.00.ВО), перед номерами телефонов (тел. 89997773737), в нумерации (1.1, 1.2, …), в составе сокращения ФИО (А.А. Иванов) и в инициальных аббревиатурах (R.I.S.K.). 
\par\textbf{Второй этап} разбивается на две параллельных части:\emph{поиск мест введения аббревиатур и поиск аббревиатур без расшифровок} в предложениях.
Поиск \emph{мест введения аббревиатур} заключается в анализе предложения на предмет наличия аббревиатуры и соответствующей расшифровки. В случае успеха информация о расшифровке и соответствующей аббревиатуре заносится в базу данных. Одна аббревиатура вводится в тексте только один раз.
\par\emph{Введения аббревиатур} имеют определенную структуру, которая задается формулой (таблица 1) [3]. Возможны ситуации, когда при введении аббревиатуры в скобках может присутствовать текст, который не относится ни к расшифровке, ни к аббревиатуре.
(таблица~1).
\begin{table}[!h]
\let\PBS=\PreserveBackslash
\renewcommand{\multirowsetup}{\centering}
\centering \setlength{\extrarowheight}{4pt} %\vspace{-4mm}
\begin{tabular}{|>{\PBS\raggedright}m{7cm} |>{\PBS\raggedright }m{7cm} |}
\multicolumn{2}{c}{Таблица 1 --- Типовые формулы введения аббревиатур
}\\
\hline \hspace{1.5cm} \textbf{Формула введения} & \hspace{1.8cm} \textbf{Примеры}0

\\
\hlineрасшифровка (аббревиатура)
 & Специальное программное обеспечение (СПО), система (С)
\\
\hline аббревиатура (расшифровка)
& АС (автоматическая сигнализация), СО (сигнал ожидания)\\
\hline
расшифровка аббревиатура
& Процессор преобразования матриц ППМ, двухпроцессорная система ДС\\
\hline
аббревиатура – расшифровка
& ЯМД – язык манипулирования данными, ЯУ – язык управления
\\
\hline
(расшифровка – аббревиатура)
& (Главная машина – ГМ)
\\
\hline
\end{tabular}
\end{table}
\par\emph{Поиск аббревиатур без расшифровок} заключается в определении по определенным признакам наличия аббревиатур в предложении. 
\parВ процессе поиска могут встречаться аббревиатуры, которые ранее не были введены в тексте. В связи с этим появляется необходимость поиска информации об их расшифровках по другим источникам. Для корректного сопоставления аббревиатуры и расшифровки из разных текстов, необходимо учитывать контекст аббревиатур (рисунок 3). 

\parТак как одна аббревиатура вводится в тексте, как правило, только один раз, то далее по тексту будут встречаться только аббревиатуры без расшифровок, при этом каждая выделенная аббревиатура будет однозначно соответствовать введенной ранее. Далее от предложения к предложению необходимо считывать аббревиатуры и их контекст в базу данных, после чего производить сопоставление с информацией, полученной при \emph{поиске мест введения аббревиатур}. В случае отсутствия расшифровок необходимо производить поиск по другим текстам или по словарю.
\parПринципиальная схема ВАиР в тексте приведена на рисунке 4.
\par\textbf{5.Заключение}
\parВ статье рассматривается обобщенное гиперболическое уравнение запаздывающего типа с некарлемановскими сдвигами вида

В статье могут быть рисунки, например, рисунок~\ref{claster_pic1}.
\begin{figure}[!h]
\centering
\includegraphics[scale=0.8]{class_alg_claster.eps}
\caption{Классификация алгоритмов кластеризации}
\label{claster_pic1}
\end{figure}

Использовать именно эту программную конструкцию для вставки
рисунков. Соответственно, формат рисунка {\bf{eps}}, рекомендуемое
разрешение 300 dpi. Данная картинка создана в MS Visio. \par

В Вашей статье также может присутствовать табличный материал
(таблица~1).
\begin{table}[!h]
\let\PBS=\PreserveBackslash
\renewcommand{\multirowsetup}{\centering}
\centering \setlength{\extrarowheight}{4pt} %\vspace{-4mm}
\begin{tabular}{|>{\PBS\raggedright}m{7cm} |>{\PBS\raggedright }m{7cm} |}
\multicolumn{2}{c}{Таблица 1 --- Функционалы качества разбиения}\\
\hline \hspace{1.5cm} Наименование & \hspace{1.8cm} Расчетная формула\\
\hline Сумма квадратов расстояний до центров кластеров
 & $Q_1=\sum\limits_{k=1}^{h}\sum\limits_{x_i\in C_k}d^2(x_i, \bar{X}(k))$\\
\hline Сумма попарных внутриклассовых расстояний между элементами
& $Q_2=\sum\limits_{k=1}^{h}\,\sum\limits_{x_i, x_l \in
C_k}d^2(x_i, x_l)$
$Q_3=\sum\limits_{k=1}^{h}\frac{1}{n_k}\sum\limits_{x_i, x_l \in C_k}d^2(x_i, x_l)$\\
\hline
\end{tabular}
\end{table}

При оформлении таблиц рекомендуется придерживаться данного
шаблона.

Ссылки на использованную литературу также можно считать
неотъемлемой частью статьи~\cite{StroevSP1,StroevSP2}.

\begin{thebibliography}{99}
%метки для источников слдедует формировать по первой фамилии автора настоящей статьи, например, StroevSP1, StroevSP2 и т.д.
%оформление статьи и книги с числом автором < 4
\bibitem{StroevSP1} Фамилия1~{И.\,О.}, Фамилия2~{И.\,О.},
Фамилия3~{И.\,О.} Название статьи // Название журнала. --- 2014.
--- Выпуск. --- Номер. ---~С.~0--0.
\bibitem{StroevSP2} Фамилия1~{И.\,О.}, Фамилия2~{И.\,О.},
Фамилия3~{И.\,О.} Название книги. --- 2-е изд. испр. --- М.:
Наука.
--- 2009. --- 400~с.
\end{thebibliography}

\end{document}
